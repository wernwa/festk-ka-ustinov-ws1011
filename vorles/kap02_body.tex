\chapter{Struktur der Kristalle}

Kristalle ensprechen dem Grundzustand der Festkörper. Lehre --> Kristallographie;

\underline{Symmetrie}: alle Eigenschaften eines Systems, die nach einer bestimmten
Änderung (Transformation) ``unverändert erscheinen''

\underline{Gruppentheorie}:
Symmetriegruppen: ist eine Menge aller Kongruenzabbildungen die das Objekt auf
sich selbst abbilden.





\section{Punktgruppen (min 1 Punkt fest)}
\(\rightarrow\) 32 Symmetriegruppen (Punktgruppen)
z.B Drehung um eine Drehachse

pic

Spiegelung an einer Spiegelebene


pic

Punktspiegelung

pic


1,2,3,4,6 Drehachsee;
mspiegelung;
\(\overline 1\) Punktspiegelung

pic

Inversion + Drehung

pic \(\overline 2\)

\underline{Raumgruppen} (translative Symmetrieoperationen)\(\rightarrow\) J.S. Fedorov 

Gleitspiegelebene (retlection+translation)

pic

Schraubenachse (screw symmetry)

pic


\underline{Kristallstruktur}: Gitter + Basis

Gitter als Umgebung 
\[U(\vec r)=U(\vec r +\vec R)\]
Translationsvektor (Gittervektor):
\[ \vec R = n_1\vec a+ n_2\vec b+ n_3\vec c\]
\(\vec a\),\(\vec b\),\(\vec c\) sind Basisvektoren. Gitterkonstanten: \(|\vec a|=a\),
\(|\vec b|=b\), \(|\vec c|=c\)

pic

Die Vektoren spannen eine Elementarzelle auf. Das Volumen ergibt sich aus dem Spatproduckt
\[ V=\vec a [\vec b \times \vec c] \]
Kleinstmögliche E.Z. \(\rightarrow\) primitive Elementarzelle (Einheitszelle).

\underline{Bravais-Gitter}\(\rightarrow\) besteht nur aus einer Teilchensorte.

\begin{tabular}{ccc}
&Kristallstruktur&Bravais-G.\\
Anzahl von Punkgruppen& 32(3D) 10(2D)& 7 Kristallsyst.\\
Anzahl v.Raumricht&230(3D) 17(2D)&14 Bravais G.
\end{tabular}

\underline{Kristallsysteme}: (Syngonien)
\begin{enumerate}
\item kubisches K.
\item tetragonales K.
\item rombisches K.
\item Rechtwiklig ende
\item hexagonale
\item trigonale
\item monoklines
\item triklines
\item Schiefwinklig ende
\end{enumerate}



\section{Einfache Kristallgitter} 
(\(K_z\)= Koordinationszahl, zahl der nächsten Nachbarn)
(p.V. Packungsverhältniss)

\begin{itemize}
\item \textbf{sc} \(\rightarrow\) ``simple cubic'' \(p.V.\approx 0,52\); In der natur so gut
  wie nicht zu finden

2-Atom Basis (0,0,0) oder (\(\frac 1 2\),\(\frac 1 2\),\(\frac 1 2\))
CsCl

\item \textbf{bcc}=''body centered cubic''. Kubisch raumzentrierte Gitter; 30\%
  aller Elemente; \(p.V.\approx 0,68\)

Metalle: Na,Fe,Cr,...
 (0,0,0) oder (\(\frac 1 2\),\(\frac 1 2\),\(\frac 1 2\))
\item \textbf{fcc}=''face centered cubic''=kubisch flächenzentrierte Gitter

Gitterpunkte
 (0,0,0) oder (\(\frac 1 2\),\(0\),\(\frac 1 2\)) oder (\(\frac 1 2\),\(\frac 1 2\),0)
 oder (0,\(\frac 1 2\),\(\frac 1 2\))  \(p.V.\approx 0,74\) Höchste Basisverhältniss
1 Atom Basis, z.B Metalle Cu, Ag, Au, Ni,... 30\% aller Elemente
NaCl\(\rightarrow\) fcc mit 2-Atom. Basis (0,0,0) oder (\(\frac 1 2\),\(\frac 1
2\),\(\frac 1 2\))
Diamant C\(\rightarrow\)fcc mit 2-At. Basis (0,0,0) oder (\(\frac 1 4\),\(\frac 1
4\),\(\frac 1 4\))
2 unterschiedliche Atome z.B ZnS; Strukt.v.Zinkblende; Mischung von Kovalenter
Bindung und Ionenbindung.
\end{itemize}


d) Hexagonal dichteste Kugelpackung (hcp=hexagonal close packed)
~35\% aller Elemente; \(p.V.\approx 0,74\)

pic1 (folie: Kugelpackung: hexagonal oder kubisch? hcp und fcc)

z.B. Mg, Ti, Co,...

pic2 (folie: Primitive Elementarzelle)

\section{Wigner-Seitz-Zelle}

pic3 (folie: Zur 2D Konstruktion einer Wigner-Seitz-Zelle)

E.Z. mit Gitterpunkt im Zentrum der Elementarzelle; lückenlose bedeckung der
Fläche (2D) oder Volumen (3D); Polyeder mit dem kleinsten Volumen, das den
Gitterpunkt ein schließt.

pic4 (folie: 3D Wigner-Seitz-Zellen)

Wigner-Seitz-Zelle ist wichtig für Reziprokes Gitter


