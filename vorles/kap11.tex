\documentclass[10pt,a4paper,oneside,fleqn]{report}
\usepackage{geometry}
\geometry{a4paper,left=20mm,right=20mm,top=1cm,bottom=2cm}
\usepackage[utf8]{inputenc}
%\usepackage{ngerman}
\usepackage{amsmath}                % brauche ich um dir Formel zu umrahmen.
\usepackage{amsfonts}                % brauche ich für die Mengensymbole
\usepackage{graphicx}
\setlength{\parindent}{0px}
\setlength{\mathindent}{10mm}
\usepackage{bbold}                    %brauche ich für die doppel Zahlen Darstellung (Einheitsmatrix z.B)
\usepackage[linktocpage={false}]{hyperref}


\usepackage{color}
\usepackage{titlesec} %sudo apt-get install texlive-latex-extra

\definecolor{darkblue}{rgb}{0.1,0.1,0.55}
\definecolor{darkred}{rgb}{0.55,0.2,0.2}

\titleformat{\chapter}[display]{\color{darkred}\normalfont\huge\bfseries}{\chaptertitlename\
\thechapter}{20pt}{\Huge}

\titleformat{\section}{\color{darkblue}\normalfont\Large\bfseries}{\thesection}{1em}{}
\titleformat{\subsection}{\color{darkblue}\normalfont\Large\bfseries}{\thesection}{1em}{}

% Notiz Box
\usepackage{fancybox}
\newcommand{\notiz}[1]{\vspace{5mm}\ovalbox{\begin{minipage}{1\textwidth}#1\end{minipage}}\vspace{5mm}}

\usepackage{cancel}


%\includegraphics[width=0.75\textwidth]{thepic.png}

\begin{document}
%\tableofcontents
\setcounter{chapter}{10}
\chapter{Halbleiter}

\underline{Charakteristika}:

\begin{itemize}
\item Metallischen Glanz aber kein Metall
\item Negativer Temperatur Koeffizient \(\rho\uparrow \quad T\downarrow\)
\item Photoleitfähigkeit
\item Eigenschaften können von Verunreinigungen empfindlich abhängen
\end{itemize}

 Materialien:
4.hauptgruppe: Si,Se, Ga,Teller, P, B,
Verbindungen III-V: GaAs, InSb
II-VI: ZnS,CdS
IV-IV: SiC


\underline{Elektrischer Widerstand}

Metall \(\rho = 10^{-7} \text{ bis } 10^{-8}\Omega m\)
isolator \(\rho > 10^{12}\Omega m \)
Halbleiter \(\rho = 10^{-4} \text{ bis } 10^{7}\Omega m \)
\(\exists\) Bandlücke, kleiner als bei Isolatoren
bei T=0 Halbleiter sind Isolatoren
T>0 Wahrscheinlichkeit für eine Termische Anregung
\(E_g>0,1 ... 2eV\)
\(E\propto e^{-\frac{E_g}{2kT}}\)

\underline{Intrinsische Halbleiter}: Eigenschaften werden durch Thermische anregung bestimmt
\underline{Extrinsische Halbleiter}: Eigenschaften werden durch Dotierung von Frembatomen bestimmt


\begin{enumerate}
\item[1)] Intrinsische HL

  \begin{enumerate}
  \item[a)] Bandlücke und optische Abstände Indirekter Übergang
 Impuls wird durch Phonon gewährleistet; 
    \underline{ Kristallimpulserhaltung}
Übergang hängt von  Phononenspektrum ab und daher von der Temperatur abhängig. Photon: große Energie, kleiner Impuls; Phonon: kleine Energie, großer Impuls

\underline{Direkter Übergang}

schwache Temperaturabhängigkeit (vgl \(1500nm\approx 0,8eV\)
  \item[b)] Effektive Massen von Elektronen und Löchern

 Bandkrümmung in der Nähe des Übergangs,; Parabolische Näherung:
\[E_n= E_L + \frac{\hbar^2 k^2}{2m^*}\]

mit n=Elektronen und p=Löcher. Elektronen im Leitungsband im 

\begin{tabular}{ccc}
&Transversal&Longitudinal\\
  Si & \(\frac{m^*_t}{m_e}=0,19\)& \(\frac{m^*_l}{m_e}=0,19\)\\
Ge&  \(\frac{m^*_t}{m_e}=0,082\)& \(\frac{m^*_l}{m_e}=1,57\)
\end{tabular}

Löcher im Valenzband
\begin{tabular}{ccc}
&Transversal&Longitudinal\\
  Si & \(0,16mc\)& \(0,49mc\)\\
& leicht Loch& schweres Loch
\end{tabular}

\(GaAs\)
\begin{tabular}{ccc}
&Transversal&Longitudinal\\
  Löcher & \(\frac{m^*_t}{m_e}=0,12\)& \(\frac{m^*_l}{m_e}=0,61\)\\
&leicht&schwer\\
& leicht Loch& schweres Loch
\end{tabular}
 
  \item[c)] Metall-Halbleiter Übergang

Austrittsarbeit \(\phi\) zum Vakuum. Die Austrittsarbeit bestimmti die el. Eigenschaft.

n-Dotiert: \(\phi_{HL}>\phi_{ME}\) ohmscher Kontakt
\(\phi_{HL}<\phi_{ME}\) blokierender Kontakt (Schottky-Kontakt). An der Grenzfläche ensteht eine Hochohmige Verarmungszohne. Elektronen fließen ins Metall

p-Dotiert: genau andersherum



  \end{enumerate}


\item[2)] Dotierte HL

  \begin{enumerate}
  \item[a)] Spezifischer Widerstandhängtstart von der Konzentration der Verunreinigung ab.
  \item[b)] Donatoren: liefern zusätzliche Elektronen ins Leitungsband: P, As, Sb; haben eine höhere Valenz
Akzeptoren: liefern zustzliche Löcher in Valenzband. niedrigere Valenz als das Wirtsmaterial: B,Al,Ga,In

Modell: Donator verhält sich wie ein positiv geladenes Ion mit zusätzlichen Elektron. Bohr-Radius somit größer als beim H-Atom; Bindungsenergie \(\approx 10meV\)
  \end{enumerate}

\item[3)] Inhomogene HL

  \begin{enumerate}
  \item[a)] p-n Übergang


    \begin{itemize}
    \item angleichung des chem. Potentials (\(E_F\))
    \item Verarmung freier Ladungsträger im Bereich des Übergangs durch rekombination mit Ladungsträgern von anderen Typ.
    \item geladenen Störstellen bleiben zurück, es entwickelt sich eine Raumladungszone
    \end{itemize}

  \item[b)] Schottky-Motell

Kastenförmiger Verlauf der Raumladungs-Zone; \(V(x)=\)Potentialverlauf, in y,z \(\infty\) ausgedehnt Poisson Gl:

\[\Delta V(x) = \frac{-\rho(x)}{\epsilon_0}\]

selbstkonsistenzproblem: \(\rho(x)\) hängt von \(V(x)\) und umgekehrt ab. Itaratir \(\rho(x)\rightarrow V(x)\rightarrow \rho(x)\)

Dicke der Raumladungszone \(eV_D\simeq E_g\approx 1eV,n=10^{10}\text{ bis } 10^{24}\); \(d=1\mu m \text{ bis } 10nm\); vergl. Atom-Atom  \(\epsilon \approx 10^{10} \frac{V}{m}\)

  \item[c)] Ströme in Gleichgewicht

Diffusionsstrom. El aus dem n-HL rekombinieren mit Löchern p-HL \(\Rightarrow \) Ladungs

Feldstrom: Elektronen aus dem p-HL (Minoritätsladungsträder) werden durch das E-Feild in n-HL

Im Gleichgewicht heben sie sich auf. 
  \end{enumerate}

Ph Übergang unter Spannung
\begin{itemize}
\item \(E_F+eU\) muss ausgeglichen sein
\item Durchlassrichtung U rec  die Potentialdifferenz
\item Sperrichtung Pot-Diff vergrößert
\item Diode Durchlassrichtung große lLeitfähigkeit; Sperrichtung kleine Leitfähigkeit
\end{itemize}


\end{enumerate}



\end{document}
