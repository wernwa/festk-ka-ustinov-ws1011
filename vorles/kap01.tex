\documentclass[10pt,a4paper,oneside,fleqn]{report}
\usepackage{geometry}
\geometry{a4paper,left=20mm,right=20mm,top=1cm,bottom=2cm}
\usepackage[utf8]{inputenc}
%\usepackage{ngerman}
\usepackage{amsmath}                % brauche ich um dir Formel zu umrahmen.
\usepackage{amsfonts}                % brauche ich für die Mengensymbole
\usepackage{graphicx}
\setlength{\parindent}{0px}
\setlength{\mathindent}{10mm}
\usepackage{bbold}                    %brauche ich für die doppel Zahlen Darstellung (Einheitsmatrix z.B)
\usepackage[linktocpage={false}]{hyperref}


\usepackage{color}
\usepackage{titlesec} %sudo apt-get install texlive-latex-extra

\definecolor{darkblue}{rgb}{0.1,0.1,0.55}
\definecolor{darkred}{rgb}{0.55,0.2,0.2}

\titleformat{\chapter}[display]{\color{darkred}\normalfont\huge\bfseries}{\chaptertitlename\
\thechapter}{20pt}{\Huge}

\titleformat{\section}{\color{darkblue}\normalfont\Large\bfseries}{\thesection}{1em}{}
\titleformat{\subsection}{\color{darkblue}\normalfont\Large\bfseries}{\thesection}{1em}{}

% Notiz Box
\usepackage{fancybox}
\newcommand{\notiz}[1]{\vspace{5mm}\ovalbox{\begin{minipage}{1\textwidth}#1\end{minipage}}\vspace{5mm}}

\usepackage{cancel}



%\includegraphics[width=0.75\textwidth]{thepic.png}

\begin{document}
%\tableofcontents
\setcounter{chapter}{0}
\chapter*{Vorbemerkung}

\underline{Festkörperphysik}: Aufbau und Eigenschaften fester Materialien

\underline{Festkörper}: Ansammlung von Atomkernen und Elektronen \(\approx
10^{23}\) die über elektrostatische Kräfte miteinander wechselwirken

\underline{Klassen}
\begin{enumerate}
\item Isolatoren
\item Halbleiter
\item Metalle+Supraleiter
\end{enumerate}

Fundamentale Konzepte in der Festkörperphysik:

\begin{enumerate}
\item Schrödinger-Gleichung
\item Pauli-Prinzip
\item Coulomb-WW
\item Maxwell-Gleichungen
\item Thermodynamik
\item Statistische Physik
\end{enumerate}

\boxed{Max Born} Phononen=elastische Schwingungen

\chapter{Bindungskräfte im Festkörper}
5 Grundtypen der Bindung
\begin{enumerate}
\item Fluktuationsbindung (Van-der-Waals Kraft
\item Ionenbindung (NaCl)
\item Kovalente Bindung (Diamant)
\item Metallische Bindung
\item Wasserstoffbrückenbindung
\end{enumerate}

\underline{Bindungsenergie} 
die Arbeit, die bei der Zerlegung des Festkörper in seine Bestandteile (Atome
oder Moleküle ) aufgewendet werden muss.

z.B. die Elemente der zweiten Periode des Periodensystems:
\begin{tabular}{ccccccccc}
&Li&Be&B&C(Diamant)&N&O&F&Ne\\
Bindungsenergie\(\left[\frac{eV}{Atom}\right]\)&1,6&3,3&5,8&7,4&4,9&2,6&0,8&0,02\\
Schmelztemperatur[K]&453&1560&2348&4765&63(*)&54(*)&53(*)&24
\end{tabular}

(*) Molekülkristalle \(N_2,O_2,F_2\). Aus diesem Grund behält die Flüssigkeit die
meiste Energie.


http://de.wikipedia.org/wiki/Lennard-Jones-Potential
Das Lennard Jones Potential:
Potential zwischen neutralen Atomen (oder Molekülen) mit abgeschlossener
\(e^-\)-Schale
a) der anziehende Teil \(\approx -r^{-6}\)
b) der abstoßende Teil \(\approx r^{-12}\)

\(\boxed{\phi(r)=\frac A {r^{12}}-\frac B {r^{6}}}\)

pic TODO

\underline{Van-der-Waals-Bindung}

pic TODO

\(\phi_{12}(\vec r) \propto \frac {\vec p_1\vec p_2}{r^3}-\frac{3(\vec p_1 \vec
  r)(\vec p_2 \vec r}{r^5}\)

\section{Bindungsenergie von Edelgaskristallen}

http://de.wikipedia.org/wiki/Lennard-Jones-Potential
Das Lennard Jones Potential:

 
\begin{eqnarray}
\phi(r) &=& \frac A {r^{12}}-\frac B {r^6} \\
&=&4\epsilon \left[ \left(\frac\sigma r\right)^{12}-\left(\frac\sigma r\right)^6\right]
\end{eqnarray}
das Potential minimum tritt bei \(r_0=2^{\frac 1 6}\sigma\approx 1,12\sigma\)
Bindungsenergie von N Atomen
\[ V_B=\frac 1 2 \sum_i\phi_i=\frac N 2 \phi_1=2N\epsilon\sum_{j\neq1}\left[
  \left(\frac\sigma r\right)^{12}-\left(\frac\sigma r\right)^6\right] \]
Kubisch flächenzentrierte Struktur fcc=face centered cubic \(r_{ij}=R=12;R\sqrt
2=6;2R...\)
\[ U_B=2N\epsilon\left[
  \left(\frac\sigma r\right)^{12}\cdot\underbrace{
    \left(\frac{12}{1^12}+\frac{6}{\sqrt 2^{12}}+...\right)}_{\approx 12,13}
-\left(\frac\sigma r\right)^6 \cdot\underbrace{
  \left(\frac{12}{1^12}+\frac{6}{\sqrt 2^{12}}+...\right)}_{\approx 14,45}
\right]
\]

\(R_0\approx 1,09\sigma\rightarrow\) nur für \(E_{kin}=0\) (in qm \(E_{kin}\neq
0\Rightarrow R_{0qm}>R_0\)) ein Minimum von \(U_B(R)\) beim
\(R=R_0\Rightarrow\left.\frac{dU_B}{dR}\right|_{R=R_0}=0;\left.\frac{d^2U_B}{dR^2}\right|_{R=R_0}>0\)

\begin{tabular}{c|cccc}
&Ne&Ar&Kr&Xe \\
\(\sigma(A^{\circ})\)&2,74&3,40&3,65&3,98 \\
\(\epsilon(meV)\)&3,1&10,4&14,1&20,0 \\
\(\frac {R_0} \sigma\)&1,15&1,11&1,09&1,09
\end{tabular}

\underline{Nullpunktenergie}

\end{document}
